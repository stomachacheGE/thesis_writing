%!TEX root = ThesisLKN.tex
\chapter{Conclusions and Outlooks} \label{chapter:6}

In this thesis work, we extend the BOF framework with a learned human motion model for human tracking in indoor environments. We call our filter \textit{Bayesian Occupancy Filter using Motion Pattern} (BOFMP). The human motion pattern is represented as a set of conditional probability distributions for each cell, which represents the likelihood of each exit direction conditioned on the entering directions. In this way, we decompose the \textit{global} task of modeling human motion pattern into \textit{local} tasks at cell level, which makes our motion model \textbf{place dependent}. By incorporating conditional dependency, the motion model also captures the \textbf{spatial correlation} between cells.

Our proposed BOFMP, along with its baseline method BOFMP, is evaluated on both simulated and real tracking cases. Depending on whether measurements from sensors are given, the tracking process is divided into two stages: \textit{tracking} and \textit{future prediction}. However, the future prediction stage is more interesting, since it represents the scenarios when occlusion happens. We show that our BOFMP outperforms BOFUM in prediction stage on both simulated and real data. Moreover, the BOFMP filter is further extended with spatial blurring and motion keeping, which add extra performance gain to our method. Besides human tracking, possible usages of BOFMP are demonstrated, such as predicting future occupancies and frequently occupied areas, which could be useful for applications like collision avoidance and path planing for robots.

In terms of implementations, we offer a \texttt{Python} \footnote{\url{https: / / www.python.org}} package that is complete enough to replicate all the results presented in this thesis, and also possible to be extended for future works. The main functionalities of the package include: 1) Human trajectory simulation with Astar algorithm and a complete pipeline of generating ground truth probabilities from these trajectories. 2) Neural network training with a deep learning library \texttt{Theano} \footnote{\url{http:/ / deeplearning.net/software/theano/\#}}, and training on both CPU and GPU is possible. 3) An implementation of BOFMP filter within an objected-oriented paradigm, which is able to perform, visualize and evaluate tracking on both simulated and real data.

\section{Future Work}

This thesis demonstrates that our proposed BOFMP outperforms its baseline method BOFUM in human tracking in indoor environments. However, there are still rooms for possible improvements:
\begin{my_enumerate}
\item To learn human motion pattern, a big amount of human trajectories on various maps is needed. However, due to limited time of this thesis work, we are not able to collect enough real human trajectories. As a compromise, our motion model is learned from simulated trajectories. On one hand, one of the advantages of our method is that we prove learning from simulated data can be successfully transfered to applications on real data. On the other hand, if we could collect enough real human trajectories, we would expect a higher performance gain. 
\item Our proposed motion pattern, which is represented by $P_c(V^{ex}|V^{en})$, models highest speed of 1 $cell/timestep$. However, human do not walk with a constant speed in reality. Although we relax this limitation by a technique introduced in Section \ref{sec:mm_ext}, it would be better to directly model higher speeds in the motion pattern. One possible solution is to model cell state as an Input-Output Hidden Markov Model (IOHMM), and the parameters can be learned by a generalized EM algorithm \cite{wang2014modeling}.
\item Although our motion model captures the general motion patterns in indoor environment, it assumes this motion pattern is static. In many cases, human motion pattern may depend on the time horizon. For example, during lunch time, it is likely that there are more occupancies near a door towards canteen. It will be interesting to capture the time aspect of motion model in the future work.
\item As shown in Section \ref{sec:BOFMP_implementation}, we extend the BOFMP filter with the idea of motion keeping so that it is better at summarizing past motion trend and therefore improves its future predictions. However, our current implementation suffers from the limitation that if there are multiple persons perform different motion patterns, the calculated moving average is no longer accurate. To deal with this problem, one idea is to use multiple occupancy maps at different discretization levels. By considering the occupancies on these hierarchical occupancy maps, we expect the past motion trend can be better summarized. In fact, the similar idea has been successfully adopted for modeling dynamics in environments by Temporal Occupancy Grid (TOG) \cite{arbuckle2002temporal}. However, they propose hierarchies on time horizon, while we propose hierarchies on space.
\end{my_enumerate}
